\chapter{The first appendix}

% Text for the first appendix goes here.

% \section{Appendix section}

% Text for a section in the first appendix goes here.

% test ทดสอบฟอนต์ serif ภาษาไทย

% \textsf{test ทดสอบฟอนต์ sans serif ภาษาไทย}

% \verb+test ทดสอบฟอนต์ teletype ภาษาไทย+

% \texttt{test ทดสอบฟอนต์ teletype ภาษาไทย}

% \textbf{ตัวหนา serif ภาษาไทย \textsf{sans serif ภาษาไทย} \texttt{teletype ภาษาไทย}}

% \textit{ตัวเอียง serif ภาษาไทย \textsf{sans serif ภาษาไทย} \texttt{teletype ภาษาไทย}}

% \textbf{\textit{ตัวหนาเอียง serif ภาษาไทย \textsf{sans serif ภาษาไทย} \texttt{teletype ภาษาไทย}}}

% \url{https://www.example.com/test_ทดสอบ_url}

\chapter{\ifenglish Manual\else คู่มือการใช้งานระบบ\fi}

\section{คู่มือการดาวน์โหลดโปรแกรม}

\enskip \enskip \enskip การใช้งานโปรแกรมขั้นแรก 
ผู้สนใจทดลองใช้โปรแกรมสามารถค้นหาและดาวน์โหลดโปรแกรมได้ที่ 
https://github.com/Gravitumn/ESG-classification 


\section{คู่มือการฝึกสอนโมเดลเพิ่มเติม}

\enskip \enskip \enskip ผู้ที่มีความสนใจจะฝึกสอนโมเดลเพิ่มเติมสามารถทำการเรียกใช้ฟังก์ชัน 
Train\_model ซึ่งอยู่ภายในไฟล์ Train.py ได้ โดยมี Arguments ที่สามารถส่งเข้าไปในฟังก์ชันได้ดังนี้


\begin{enumerate} 
    \item
    bert\_model เป็น argument ที่ใช้ในการระบุโมเดลที่ใช้ในการฝึกสอนโดยใช้ file path ในการระบุ หากต้องการใช้โมเดลที่ฝึกสอนโดยผู้จัดทำ สามารถใช้ไฟล์ model/albert2 ซึ่งเตรียมไว้ให้ในโฟลเดอร์ได้เลย
    แต่หากต้องการใช้โมเดลชนิดอื่นก็สามารถใส่ path ดังตัวอย่างด้านล่างได้เช่นกัน
    \item
    tokenizer เป็นการระบุ tokenizer ที่จะใช้ในการแบ่งประโยคซึ่งโดยปกติจะใช้เป็น
    tokenizer ของโมเดลนั้นๆ ดังตัวอย่างด้านล่าง
    \item
    device เป็นการระบุอุปกรณ์ที่ใช้ในการฝึกสอน ตามตัวอย่างด้านล่าง
    \item
    file\_path ของ dataset ที่เราจัดเตรียมไว้ 
    โดยไม่จำเป็นต้องแบ่ง Train/test เนื่องจาก จะมีให้ระบุเป็น Argument 
    \item
    batch\_size เป็นพารามิเตอร์ของการฝึกสอนโมเดลที่ช่วยให้โมเดลเรียนรู้ได้ดีมากขึ้นหากปรับได้อย่างเหมาะสม 
    นอกจากนี้การปรับให้มี batch\_size ที่สูงเกินไปยังส่งผลให้หน่วยความจำของคอมพิวเตอร์ไม่เพียงพออีกด้วย โดยมี default batch size = 8
    \item
    Shuffle เป็นพารามิเตอร์ที่ใช้ระบุว่าต้องการสับเปลี่ยนการเรียงลำดับของ dataset 
    หรือไม่ซึ่งมีค่า default เป็น True และแนะนำให้ตั้งค่าเป็น True เสมอ
    \item
    lr หรือ learning\_rate เป็นพารามิเตอร์ที่ใช้ในการระบุ 
    learning rate ในการเรียนรู้ของคอมพิวเตอร์ มีค่า default = 1e-5
    \item
    num\_epochs เป็นพารามิเตอร์ที่ใช้ในการระบุว่าโมเดลจะได้รับการฝึกสอนเป็นจำนวนกี่ครั้ง 
    โดยมีค่า default = 100
    \item
    T\_0 เป็นพารามิเตอร์สำหรับ Cosine annealing warm restart โดย T\_0 จะเป็นตัวกำหนดว่าจะให้โมเดลทำการเรียนรู้กี่ epoch ก่อนจะเกิดการ restart ซึ่งในระหว่างนั้นจะมีการปรับ learning rate ให้ต่ำลงและเมื่อเกิดการ restart จะทำให้ learning rate 
    กลับมาเท่ากับค่าเริ่มต้น ช่วยให้โมเดลสามารถเรียนรู้ต่อไปได้ มีค่า default= 2
    \item
    T\_mult เป็นพารามิเตอร์สำหรับ Cosine annealing warm restart เป็นตัวคูณซึ่งใช้ในการกำหนดจำนวนรอบก่อนจะเริ่ม restart อีกครั้งหนึ่ง ยกตัวอย่างเช่นหาก T\_mult = 2 และ T\_0 = 10 เมื่อเกิดการ restart จะต้องรออีก 20 ครั้งเพื่อจะ restart และหลังจากนั้นจะเพิ่มเป็น 40 ครั้ง 
    และ 80 ครั้งจนกว่าจะพบเงื่อนไขการจบการฝึกสอน โดยมีค่า default = 2
    \item
    eta\_min เป็นพารามิเตอร์สำหรับกำหนดว่า learning\_rate สามารถลดลงต่ำสุดได้มากเพียงใด 
    เพื่อไม่ให้ learning\_rate ลดต่ำจนเกินไป โดยมีค่า default = 1e-6
    \item
    model\_save\_path เป็น argument สำหรับระบุ file path ที่จะใช้ในการบันทึกโมเดลเมื่อจบการฝึกสอน 
    โดยมี default path เป็น /model/trained\_model
    \item
    early\_stop\_epoch เป็น argument สำหรับระบุจำนวนรอบสูงสุดที่ต้องการให้โมเดลทำการฝึกสอน
    โดยมีค่าตั้งต้น = 8 รอบ
    \item
    return\_result เป็น argument สำหรับระบุว่าผู้ใช้งานต้องการผลลัพธ์จากการฝึกสอนหรือไม่ โดยผลลัพธ์เหล่านี้จะเป็น confusion matrix ที่แสดงการทำนายประโยคของโมเดล 
    และกราฟ accuracy ในแต่ละ epoch โดยมีค่า default = False
    \item
    test\_size เป็น argument สำหรับการทำ train\-test\-split 
    โดยมีค่าเริ่มต้น = 0.1 ซึ่งเป็นการระบุว่าใช้ test set 10\% ของ dataset
    \item
    random\_state เป็น argument สำหรับใช้ระบุว่าจะทำการ train\-test\-split 
    ที่ random\_state เท่าใด โดยมีค่าตั้งต้นเป็น 69
\end{enumerate}



