\chapter{\ifproject%
\ifenglish Project Structure and Methodology\else โครงสร้างและขั้นตอนการทำงาน\fi
\else%
\ifenglish Project Structure\else โครงสร้างของโครงงาน\fi
\fi
}

\makeatletter

% \renewcommand\section{\@startsection {section}{1}{\z@}%
%                                    {13.5ex \@plus -1ex \@minus -.2ex}%
%                                    {2.3ex \@plus.2ex}%
%                                    {\normalfont\large\bfseries}}

\makeatother
%\vspace{2ex}
% \titleformat{\section}{\normalfont\bfseries}{\thesection}{1em}{}
% \titlespacing*{\section}{0pt}{10ex}{0pt}

\section{การเตรียมข้อมูล}

% \begin{figure}
% \begin{center}
% \includegraphics{800px-Briny_Beach.jpg}
% \end{center}
% \caption[Poem]{The Walrus and the Carpenter}
% \label{fig:walrus}
% \end{figure}

\subsection{ดาวน์โหลดข้อมูล}

\enskip \enskip \enskip เก็บเอกสารรายงานประจำปีย้อนหลังไม่เกิน 5 ปี จากเว็ปไซต์ของตลาดหลักทรัพย์แห่งประเทศไทย
โดยเอกสารจะอยู่ในรูปแบบไฟล์ PDF เพื่อเตรียมสำหรับจัดทำข้อมูลประโยคที่เกี่ยวข้องกับการดำเนินงาน
ที่ให้ความสำคัญกับ สิ่งแวดล้อม(Environment), สังคม(Social) และธรรมาภิบาล(Governance) 

\subsection{แบ่งรายงานประจำปีเป็นประโยค}

\enskip \enskip \enskip สร้างโปรแกรมแบ่งประโยคที่เมื่อนำรายงายประจำปีเข้าไปแล้วโปรแกรมจะตัดประโยคที่ไม่สมบูรณ์หรือพวกของที่ไม่จำเป็น 
เช่น ตาราง และหัวข้อ ออกไปและคัดมาให้แค่ประโยคที่สามารถใช้กับโมเดลได้ออกมาเป็นผลลัพธ์


\subsection{เลือกข้อมูลที่จะนำไปใช้พัฒนาโปรแกรม}

\enskip \enskip \enskip สุ่มอ่านประโยคในรายงานประจำปีของบริษัทที่เป็นผลลัพธ์ของโปรแกรมแบ่งประโยคที่เตรียมเอาไว้เพื่อเตรียมข้อมูลประโยคประเภทสิ่งแวดล้อม(Environment), 
สังคม(Social), ธรรมาภิบาล(Governance) และประโยคที่ไม่เข้าประเภทใดเลย(Neutral)
  

\subsection{การทำความสะอาดข้อมูล}

\enskip \enskip \enskip นำประโยคที่เลือกมาทำการทำความสะอาดข้อมูลก่อนนำไปใช้กับโมเดล 
โดยวิธีการที่ใช้ทำความสะอาดข้อมูลคือ การตัดเครื่องหมายหรือสัญลักษณ์พิเศษต่างๆออก, การตัดurlออก, การเปลี่ยนตัวอักษรทั้งหมดเป็นตัวพิมเล็ก และการลบคำหยุดออก


\section{การพัฒนาโมเดลการเรียนรู้ของเครื่อง}

\enskip \enskip \enskip พัฒนาโมเดลการเรียนรู้ของเครื่องเพื่อนำมาทำนายชนิดของประโยคว่าเป็นหัวข้อใดในESGหรือไม่เกี่ยวข้องกับหัวข้อใดเลย 
โดยทดลองกับโมเดลหลายแบบคือ LSTM, Bert, Distilbert, Albert-V2, Electar-base, Electar-small และRobertaแล้วเลือกโมเดลที่ดีที่สุดโดยวัดประสิทธิภาพจากค่า Accuracy


\section{การพัฒนาโปรแกรมอ่านรายงานประจำปีของบริษัท}

\enskip \enskip \enskip นำโมเดลการเรียนรู้ของเครื่องที่ดีที่สุดที่พัฒนามาได้มาสร้างโปรแกรมนับจำนวนประโยคแต่ละชนิดในรายงานประจำปีที่ใส่เข้าไปเป็นอินพุต
เพื่อดูว่ารายงานประจำปีที่ใส่เข้าไปนั้นมีการพูดถึงประเกี่ยวกับสิ่งแวดล้อม, 
สังคม และธรรมาภิบาลมากน้อยเพียงใดและพูดประโยคที่ไม่เกี่ยวข้องกับประเด็นใดเลยในESGมากน้อยเพียงใด



