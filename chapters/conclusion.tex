\chapter{\ifenglish Conclusions and Discussions\else บทสรุปและข้อเสนอแนะ\fi}

\section{\ifenglish Conclusions\else สรุปผล\fi}

ในการทำโครงงานนี้สามารถพัฒนาระบบที่ใช้ในการจำแนกประเภทประโยคในรายงานประจำปีอัตโนมัติ
โดยระบบนี้มีความแม่นยำถึง 92.07 \% และประโยคที่ทำนายผิดก็มักเป็นประโยคที่จำแนกได้ยากจริงๆแม้จะให้มนุษย์มาจำแนกเองและได้ทำการทดลองนำไปใช้กับรายงานประจำปีทั้งเล่มในหลายๆปีและหลายๆบริษัทแล้วก็ได้พบว่ามีแนวโน้มที่หลายๆบริษัทมักมีเหมือนกันดังที่เห็นในผลลัพธ์จากหัวข้อ4.2



\section{\ifenglish Challenges\else ปัญหาที่พบและแนวทางการแก้ไข\fi}

ในการทำโครงงานนี้ พบว่าเกิดปัญหาหลักๆ ดังนี้


\begin{enumerate} 
    \item
    การหาวิธีในการจำแนกประโยคในรายงานประจำปีของแต่ละบริษัทเพื่อใช้สำหรับพัฒนาโมเดลจำแนกประโยค
    \item
    หาวิธีเลือกดึงประโยคที่สมบูรณ์เหมาะสำหรับในการใช้กับโมเดลจำแนกประโยคโดยอัตโนมัติ
\end{enumerate}

\section{\ifenglish%
Suggestions and further improvements
\else%
ข้อเสนอแนะและแนวทางการพัฒนาต่อ
\fi
}

ข้อเสนอแนะเพื่อพัฒนาโครงงานนี้ต่อไป มีดังนี้


\begin{enumerate} 
    \item
    ทำให้โปรแกรมสามารถรับมือกับการจำแนกประโยคที่มีหลายประเด็นของ ESG ในหนึ่งประโยคได้
    \item
    พัฒนาระบบที่ใช้กับรายงานฉบับภาษาไทยได้
\end{enumerate}
