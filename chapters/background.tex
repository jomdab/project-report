\chapter{\ifenglish Background Knowledge and Theory\else ทฤษฎีที่เกี่ยวข้อง\fi}

การทำโครงงาน เริ่มต้นด้วยการศึกษาค้นคว้า ทฤษฎีที่เกี่ยวข้อง หรือ งานวิจัย/โครงงาน ที่เคยมีผู้นำเสนอไว้แล้ว ซึ่งเนื้อหาในบทนี้ก็จะเกี่ยวกับการอธิบายถึงสิ่งที่เกี่ยวข้องกับโครงงาน เพื่อให้ผู้อ่านเข้าใจเนื้อหาในบทถัดๆ ไปได้ง่ายขึ้น

\section{Data Cleaning}


\enskip \enskip \enskip กระบวนการตรวจสอบ สะสาง แก้ไข หรือจัดรูปแบบข้อมูลให้อยู่ในสภาพที่พร้อมใช้งานที่สุด
รวมไปถึงคัดกรองข้อมูลที่ไม่ถูกต้องหรือไม่จำเป็นออกไปจากชุดข้อมูลที่จะใช้วิเคราะห์หรือประมวลผล 
เพื่อให้ชุดข้อมูลที่จะใช้มีความสมบูรณ์ มีคุณภาพ พร้อมนำไปวิเคราะห์และใช้ประโยชน์


\section{Bidirectional Encoder Representations from Transformers (BERT)}
\enskip \enskip \enskip Bert มีชื่อเต็มว่า Bidirectional Encoder Representations from Transformersคือ 
โมเดลที่ต่อยอดมาจากโมเดลที่เรียกว่า Transformers ซึ่งถูกออกแบบมาให้เลือกใช้เฉพาะส่วนที่เป็น 
encoder ทำหน้าที่ในการแปลงคำในประโยคให้กลายเป็นเวกเตอร์ 
จากนั้นจึงใช้วิธีการฝึกโมเดลในรูปแบบที่ต่างออกไปจากโมเดลทางภาษาอื่นๆ ซึ่งการฝึกของ BERT จะแบ่งออกเป็นสองส่วนได้แก่

\subsection{Masked Language Model}

\enskip \enskip \enskip เป็นการฝึกโมเดลโดยคำในประโยคที่ป้อนเข้ามาเป็นอินพุตของระบบจะถูกลบออกไปบางส่วนและเรียกคำที่ถูกลบออกไปนี้ว่า 
(Masked words) โมเดลจะต้องพยายามเติมคำที่หายไปเหล่านี้ให้ถูกต้องซึ่งหากโมเดลจะสามารถเติมคำได้ถูกต้อง
การเรียนรู้แบบนี้ช่วยให้โมเดลสามารถเรียนรู้ความสัมพันธ์และบริบทของคำในประโยคได้ดีโดยข้อมูลที่นำมาใช้ในการฝึกโมเดลเป็นคลังข้อมูลทางภาษาที่มีขนาดใหญ่ 
เช่น Wikipedia

\subsection{Fine Tuning on Specific Tasks}

\enskip \enskip \enskip เป็นการเพิ่ม Layer พิเศษเข้าไปในชั้นเอาท์พุตของโมเดล 
โดย Layer พิเศษเหล่านี้จะสามารถทำให้โมเดลมีความสามารถอื่นเพิ่มเติม 
เช่น สามารถประมวลผลได้ว่าประโยคที่รับเข้ามามีใจความที่เป็นแง่บวกหรือแง่ลบ,
ความสามารถในการตอบคำถาม รวมถึงความสามารถในการแปลภาษา
จากข้อมูลข้างต้น ด้วยความสามารถที่มากกว่าโมเดลทางภาษาแบบอื่นและความสะดวกสบายในการปรับใช้งานได้หลากหลายรูปแบบ 
ทำให้โมเดลทางภาษาแบบ BERT เป็นโมเดลทางภาษาที่ได้รับความนิยมเป็นอย่างมากและถูกใช้งานในหลายภาคส่วน 
เช่น ระดับอุตสาหกรรมและในเชิงวิชาการ

% \subsubsection{Subsubsection 1 heading goes here}
% Subsubsection 1 text

% \subsubsection{Subsubsection 2 heading goes here}
% Subsubsection 2 text

\section{Recurrent Neural Network (RNN)}
\enskip \enskip \enskip เแบบหนึ่งที่ออกแบบมาแก้ปัญหาสำหรับงานที่ข้อมูลมีลำดับ
Sequence โดยใช้หลักการ Feed สถานะภายในของโมเดลกลับมาเป็น Input ใหม่ คู่กับ Inputปกติ เรียกว่า
Hidden State, Internal State, Memory ช่วยให้โมเดลรู้จำ Pattern ของลำดับ
Input Sequence ได้

\subsection{Long-Short Term Memory (LSTM)}

\enskip \enskip \enskip จากปัญหาที่เกิดขึ้นใน RNNs เกี่ยวกับค่า gradient ที่มีค่าน้อยลงจากการทำงานของ backpropagation
จึงได้มีการคิดค้น machine learning ตัวใหม่ที่ใช้หลักการคล้าย ๆ เดิม แต่เปลี่ยนตัวฟังก์ชันด้านในให้มีความเสถียรและมีประสิทธิภาพมากขึ้น 
ซึ่งนั่นก็คือ Long Short-Term Memoryหรือเรียกย่อๆว่า LSTMs สิ่งที่โดดเด่นขึ้นมานั้นก็คือการที่มันสามารถเลือกได้ว่าข้อมูลไหนที่ควรจะจดจำ
ข้อมูลไหนที่ควรจะกำจัดทิ้งออกไปผ่านการลืมของสถานะใน node นั้น ๆ

\section{ELECTRA}

\enskip \enskip \enskip โมเดล ELECTRA เป็นโมเดลทางภาษาที่นำเสนอเป็นวิธีการ pretraining ใหม่ๆ โดยการฝึกโมเดล transformer สองตัว คือ generator และ discriminator 
โดยที generator มีบทบาทในการแทนที่โทเค็นในลำดับและจึงถูกฝึกเป็น masked language model ในขณะที่ discriminator 
ซึ่งเป็นโมเดลที่เราสนใจพยายามระบุว่าโทเค็นไหนถูกแทนที่โดย generator ในลำดับนั้นๆ
% \begin{center}
% \begin{minipage}{2em}
% juxtaposition
% \end{minipage}
% \end{center}

\section{การลงทุนอย่างยั่งยืน}

\enskip \enskip \enskip หมายถึง แนวคิดการลงทุนที่คำนึงถึงการดำเนินงานด้านสิ่งแวดล้อม 
สังคม และบรรษัทภิบาลของธุรกิจประกอบการพิจารณาตัดสินใจลงทุนควบคู่ไปกับการวิเคราะห์ข้อมูลทางการเงินของธุรกิจ 
เพื่อสร้างผลตอบแทนในระยะยาวและสร้างผลกระทบเชิงบวกหรือลดผลกระทบเชิงลบต่อสังคมและสิ่งแวดล้อม

\section{\ifenglish%
\ifcpe CPE \else ISNE \fi knowledge used, applied, or integrated in this project
\else%
ความรู้ตามหลักสูตรซึ่งถูกนำมาใช้หรือบูรณาการในโครงงาน
\fi
}

\enskip \enskip \enskip จากหลักสูตรที่ได้เรียนทั้งหมดที่ผ่านมาทำให้ได้ความรู้จากวิชา 261448 หรือ Data Mining For CPE ที่มีความรู้พื้นฐานเกี่ยวกับการเรียนรู้ของเครื่อง และวิชา
261456 หรือ Intro Computer Intelligence For CPE ที่มีความรู้เกี่ยวกับโครงข่ายประสาทเทียมและวิชา
261459 หรือ Deep Learning ที่่มีความรู้เกี่ยวกับพื้นฐานเกี่ยวกับการเรียนรู้เชิงลึกและวิชา
261499 หรือ Natural Language Processing ที่่มีความรู้เกี่ยวกับพื้นฐานเกี่ยวกับการประมวลผลภาษาธรรมชาติ นำมาใช้เป็นแนวคิดในการพัฒนาตัวโครงงานนี้


\section{\ifenglish%
Extracurricular knowledge used, applied, or integrated in this project
\else%
ความรู้นอกหลักสูตรซึ่งถูกนำมาใช้หรือบูรณาการในโครงงาน
\fi
}

\enskip \enskip \enskip ความรู้ในทางการเงินในการอ่านรายงานประจำปีและวิธีจำแนกประโยคว่าประโยคนั้นๆมีหัวข้อใดบ้างใน ESG เพื่อหาข้อมูลมาใช้ในการพัฒนาโมเดล
