\chapter{\ifenglish Introduction\else บทนำ\fi}

\section{\ifenglish Project rationale\else ที่มาของโครงงาน\fi}

    \enskip \enskip \enskip ในปัจจุบันการลงทุนในหุ้นนั้นถือเป็นหนึ่งในการลงทุนที่ดีที่สุด เนื่องจากการลงทุนในหุ้นนั้นมีข้อดีมากมาย
    เช่น การมีสภาพคล่องและผลตอบแทนที่สูง ความสามารถในการสร้างกระแสเงินสดจากเงินปันผล
    นอกจากนั้นถ้าถือในระยะยาวพอจะทำให้โอกาสในการขาดทุนนั้นมีน้อยมาก
    แต่การลงทุนในหุ้นก็ไม่ได้เหมาะกับคนทุกคน
    เนื่องจากหุ้นนั้นเป็นสินทรัพย์ที่มีความผันผวนของราคาสูง
    ดังนั้นคนที่รับความเสี่ยงได้น้อยก็จะไม่เหมาะกับการลงทุนประเภทนี้


    การลงทุนในหุ้นนั้นมีแนวทางมากมาย เช่น การลงทุนในหุ้นเติบโต 
    การลงทุนในหุ้นคุณค่าหรือการลงทุนโดยเน้นลงทุนในหุ้นที่ให้เงินปันผลที่สูง 
    โดยในโครงงานนี้จะนำเสนอแนวทางการลงทุนที่มีชื่อว่า 
    การลงทุนอย่างยั่งยืน (Sustainable Investment) โดยการลงทุนอย่างยั่งยืนนี้
    จะมุ่งเน้นการลงทุนไปยังบริษัทที่ให้ความสำคัญกับสามด้าน คือ สิ่งแวดล้อม (Environment), สังคม(Social)
    และธรรมาภิบาล (Governance) โดยจะมีการให้คะแนนทางด้านความยั่งยืนที่มีชื่อว่าคะแนน ESG

    โครงงานฉบับนี้มีจุดประสงค์สองอย่าง คือ สร้างระบบที่สามารถบ่งบอกได้ว่าบริษัทแต่ละบริษัทมีประโยคที่พูดถึงเกี่ยวกับประเด็นที่เกี่ยวข้องกับ ESG
    อยู่มากน้อยเพียงใดจากการให้ระบบอ่านรายงานประจำปีของบริษัทต่างๆ และพัฒนาโมเดลด้านภาษาธรรมชาติที่ดีที่สุดจากโมเดลด้านภาษาธรรมชาติหลายๆโมเดล
    เพื่อจำแนกว่าประโยคแต่ละประโยคในรางงานประจำปีพูดถึงหัวข้อใดใน ESG หรือไม่พูดถึงหัวข้อใดเลย
\section{\ifenglish Objectives\else วัตถุประสงค์ของโครงงาน\fi}
\begin{enumerate} 
    \item
    พัฒนาโปรแกรมเพื่อวิเคราะห์ประโยคว่าประโยคที่ปรากฏในรายงานประจำปีเป็นชนิด
    สิ่งแวดล้อม(Environment), สังคม (Social), ธรรมาภิบาล (Governance) 
    หรือประโยคที่ไม่เข้าชนิดใดเลย (Neutral) 
    โดยใช้โมเดลด้านภาษาธรรมชาติและค้นหาว่าโมเดลด้านภาษาธรรมชาติชนิดใดที่นำมาใช้กับงานนี้ได้ดีที่สุด
    \item
    พัฒนาโปรแกรมเพื่อวิเคราะห์รายงานประจำปีเพื่อหาว่ารายงานประจำปีที่วิเคราะห์มีประโยคชนิดสิ่งแวดล้อม(Environment),
    สังคม (Social), ธรรมาภิบาล (Governance) 
    และประโยคที่ไม่เข้าชนิดใดเลย (Neutral) อยู่มากน้อยเพียงใดโดยใช้โปรแกรมวิเคราะห์วิเคราะห์ประโยคที่พัฒนาจากโมเดลด้านภาษาธรรมชาติ
\end{enumerate}

\section{\ifenglish Project scope\else ขอบเขตของโครงงาน\fi}

\begin{enumerate} 
    \item
    ใช้ข้อมูลจากรายงานประจำปีที่เผยแพร่ในเว็บไซต์ของตลาดหลักทรัพย์แห่งประเทศไทยเท่านั้น
    \item
    ใช้รายงานประจำปีฉบับภาษาอังกฤษเท่านั้น
\end{enumerate}

% \subsection{\ifenglish Hardware scope\else ขอบเขตด้านฮาร์ดแวร์\fi}

% \subsection{\ifenglish Software scope\else ขอบเขตด้านซอฟต์แวร์\fi}

\section{\ifenglish Expected outcomes\else ประโยชน์ที่ได้รับ\fi}

    \enskip \enskip \enskip เนื่องจากข้อมูลเกี่ยวการพูดถึงประเด็นที่เกี่ยวข้องกับESGของบริษัทต่างๆมีการเปิดเผยข้อมูลที่น้อยมาก
    โดยถ้าอยากได้ข้อมูลที่มากขึ้นก็จะมีค่าใช้จ่ายเพิ่มเติมในการเปิดเผยข้อมูล 
    ทางผู้จัดทำจึงพัฒนาโปรแกรมขึ้นมาเพื่ออ่านรายงานประจำปีของบริษัทเพื่อดูว่าบริษัทนั้นมีการพูดถึงเกี่ยวกับประเด็นESGมากน้อยเพียงใด
    เพื่อที่จะไม่ต้องใช้แรงงานมนุษย์ในการหาข้อมูลจากรายงานประจำปี
    ดังนั้นจึงทำให้ต้นทุนในส่วนนี้หายไป 
    ข้อมูลนี้จึงสามารถเปิดเผยข้อมูลทั้งหมดได้โดยไม่มีค่าใช้จ่ายใด ๆ

\section{\ifenglish Technology and tools\else เทคโนโลยีและเครื่องมือที่ใช้\fi}

\subsection{\ifenglish Hardware technology\else เทคโนโลยีด้านฮาร์ดแวร์\fi}

\enskip \enskip \enskip Notebook Acer Nitro 7 AN715-51 และ Notebook Acer Nitro 5 AN515-51 สำหรับงานทั้งหมดในโครงงานนี้

\subsection{\ifenglish Software technology\else เทคโนโลยีด้านซอฟต์แวร์\fi}

\enskip \enskip \enskip ใช้Pythonในการเขียนโปรแกรมและใช้libraryดังนี้ Pandas ,NumPy ,Scikit Learn ,Tensorflow ,Keras ,Transformers ,Spacy ,PyTorch ,PyMuPDF ,Nltk

\section{\ifenglish Project plan\else แผนการดำเนินงาน\fi}

\begin{plan}{10}{2023}{3}{2024}
    \planitem{10}{2023}{10}{2023}{ศึกษาค้นคว้าข้อมูล}
    \planitem{11}{2023}{1}{2024}{เตรียมข้อมูลสำหรับพัฒนาโมเดลทางด้านภาษา}
    \planitem{12}{2023}{2}{2024}{พัฒนาโปรแกรมจำแนกประโยคและทดลองหาโมเดลทางด้านภาษาที่ดีที่สุด}
    \planitem{3}{2024}{3}{2024}{พัฒนาโปรแกรมอ่านรายงานประจำปีของบริษัท}
\end{plan}

\section{\ifenglish Roles and responsibilities\else บทบาทและความรับผิดชอบ\fi}
\enskip \enskip \enskip ผู้จัดทำทั้งสองคนช่วยกันทำงานทั้งหมดทุกส่วน แต่จะแบ่งงานหลักๆที่แต่ละคนได้ทำเป็นส่วนใหญ่ได้ดังนี้
นายสุภาค ไชยเนตรเกษม: รับผิดชอบหน้าที่ในการใช้การประมวลผลภาษาธรรมชาติเพื่อพัฒนาโปรแกรมจำแนกประโยคและทดลองเพื่อหาโมเดลทางด้านภาษาที่ดีที่สุด และเตรียมข้อมูลสำหรับพัฒนาโมเดลทางด้านภาษา
นายธนิสร ไชยวุฒิ: รับผิดชอบหน้าที่ในการเตรียมข้อมูลสำหรับพัฒนาโมเดลทางด้านภาษา ศึกษาหาข้อมูลอ้างอิงที่จะนำมาใช้ในงานส่วนต่างๆ และพัฒนาโปรแกรมอ่านรายงานประจำปีของบริษัท

\section{\ifenglish%
Impacts of this project on society, health, safety, legal, and cultural issues
\else%
ผลกระทบด้านสังคม สุขภาพ ความปลอดภัย กฎหมาย และวัฒนธรรม
\fi}

\enskip \enskip \enskip ผลกระทบทางด้านสังคม: ระบบที่เป็นผลลัพธ์ของโครงงานนี้จะสามารถใช้เพื่อเป็นตัวช่วยสำหรับนักลงทุนทั่วไปที่สนใจในการลงทุนเชิงยั่งยืน 
เนื่องจากในปัจจุบันนี้ข้อมูลเกี่ยวกับผลการดำเนินงานทางด้าน ESG ของบริษัทยังมีไม่มากนัก 
จึงหวังว่าระบบที่ถูกพัฒนามานี้จะสามารถช่วยได้ไม่มากก็น้อย


ผลกระทบทางด้านสุขภาพ: ตัวระบบที่ถูกพัฒนาขึ้นมานั้นไม่ส่งผลกระทบต่อสุขภาพแต่อย่างใด


ผลกระทบทางด้านกฎหมาย: ข้อมูลที่ใช้ในโครงงานนี้รวมไปถึงซอฟต์แวร์ที่ใช้ในการพัฒนานั้นทั้งหมดล้วนเป็นสิ่งที่สามารถหาได้ทั่วไปโดยไม่ละเมิดลิขสิทธิ์ 
ดังนั้นการทำโครงงานนี้จึงไม่มีผลกระทบด้านกฎหมายอย่างแน่นอน


ผลกระทบทางด้านวัฒนธรรม: ตัวระบบที่ถูกพัฒนาขึ้นมานั้นไม่มีผลกระทบทางด้านวัฒนธรรมแต่อย่างใด
